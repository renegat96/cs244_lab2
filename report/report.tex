% ------------------------------------------------------------------------------
% Preamble
% ------------------------------------------------------------------------------
 
\documentclass[12pt]{article}
 
\usepackage[margin=1in]{geometry} 
\usepackage{amsmath,amsthm,amssymb,enumitem,MnSymbol}
 
\newcommand{\N}{\mathbb{N}}
\newcommand{\Z}{\mathbb{Z}}

\renewcommand\qedsymbol{$\blacksquare$}
 
\newenvironment{theorem}[2][Theorem]{\begin{trivlist}
\item[\hskip \labelsep {\bfseries #1}\hskip \labelsep {\bfseries #2.}]}{\end{trivlist}}
\newenvironment{lemma}[2][Lemma]{\begin{trivlist}
\item[\hskip \labelsep {\bfseries #1}\hskip \labelsep {\bfseries #2.}]}{\end{trivlist}}
\newenvironment{exercise}[2][Exercise]{\begin{trivlist}
\item[\hskip \labelsep {\bfseries #1}\hskip \labelsep {\bfseries #2.}]}{\end{trivlist}}
\newenvironment{problem}[2][Problem]{\begin{trivlist}
\item[\hskip \labelsep {\bfseries #1}\hskip \labelsep {\bfseries #2.}]}{\end{trivlist}}
\newenvironment{question}[2][Question]{\begin{trivlist}
\item[\hskip \labelsep {\bfseries #1}\hskip \labelsep {\bfseries #2.}]}{\end{trivlist}}
\newenvironment{corollary}[2][Corollary]{\begin{trivlist}
\item[\hskip \labelsep {\bfseries #1}\hskip \labelsep {\bfseries #2.}]}{\end{trivlist}}

\newlist{pcases}{enumerate}{1}
\setlist[pcases]{
  label=\underline{Case~\arabic*:}\protect\thiscase.~,
  ref=\arabic*,
  align=left,
  labelsep=0pt,
  leftmargin=0pt,
  labelwidth=0pt,
  parsep=0pt
}
\newcommand{\case}[1][]{%
  \if\relax\detokenize{#1}\relax
    \def\thiscase{}%
  \else
    \def\thiscase{~#1}%
  \fi
  \item
}

% ------------------------------------------------------------------------------
% Document
% ------------------------------------------------------------------------------
 
\begin{document}
 
\title{CS244, Assignment 2}
\author{Hristo Stoyanov\\
stoyanov@stanford.edu\\
\\
Petar Penkov\\
ppenkov@stanford.edu}
 
\maketitle

\section*{Introduction}

\section*{Part A}

The first step of our experiment was to compare the performance of fixed-window
congestion control algorithms while ranging the window size. The values explored
for the congestion window size were the default value of 50 packets and all
powers of 2 from 1 to 64 packets, inclusive. We fixed the timeout to the default
value of 1 second. We found that all values give consistent results with small
noise through multiple runs. Delay and throughput were low for small windows and
big for large windows. Furthermore, the power score improved while increasing
the window size to 16, up to a maximum of 12.21, and it then started to
gradually decline. The results of our experiment are summarized on Fig. 1.

\section*{Part B}

REF: V. Jacobson and M. J. Karels, Congestion Avoidance and Control, SIGCOMM 1988.

To capture the variability of link bandwidth, we implemented a simple AIMD
scheme.  Initially we left the timeout fixed to a hardcoded value of 200
milliseconds, but later changed it to be a function of an estimated round-trip
time via a low pass filter, where R is the current estimate and M is the new RTT
measurement on every ACK. [REF]

$${RTT} \leftarrow \alpha {RTT} + \left(1 - \alpha\right)M$$

From this estimate of the current RTT we compute the timeout interval to ßR. We
used the suggested parameters $\alpha = 0.9$ and $\beta = 2$. While the
performance of the scheme with a fixed timeout value was consistent [SEE GRAPH
COLOR], there was a significant bufferbloat in the  latter scheme and it was
deemed inadequate.

\section*{Part C}

Our first implementation of a delay-triggered congestion control scheme had two
fixed threshold values. The implementation used a weighted moving average to
estimate the current round-trip-time. If the RTT was below 80ms, the congestion
window is increased, and if the RTT estimate is above 100ms, the window is
decreased.

We experimented with several values of the thresholds, as well as different
values for the increase and decrease of the windows.

\subsection*{Part C1: changing the thresholds}
\subsection*{Part C2: changing the increase/decrease functions}

\section*{Part D}

\subsection*{PART D1: AIMDSoph}

For our use case we found the simple AIMD scheme lacking in two ways. First, the
timeout value of 200ms was hardcoded and would be unreasonable in situations
when just the propagation delay is longer than that. Instead, we set the timeout
to a multiple of the minimum measured round-trip time $RT_{min}$. We found that
a delay of $2*RT_{min}$ works well as it implies queuing delay of one minimum
RTT which is enough to give us feedback while preventing bufferbloat. Second, it
does not account for very small queuing delay. In this case the sender can
probably send more packets. To address this we keep a running estimate of the
RTT via a low pass filter. [REF]. 

$${RTT} \leftarrow \alpha {RTT} + \left(1 - \alpha\right)M$$

Here R denotes the current estimate and M is the measurement on the new ACK. We
use the suggested parameter α = 0.9 to have conservative changes to the current
estimate. For every ACK that maintains the running estimate drops below
$1.5*RT_{min}$, we do another step of additive increase, effectively doubling
the additive increase gain temporarily. Furthermore, to keep the algorithm more
optimistic, we decrease the multiplicative decrease factor from 2 to 1.5 and
double the additive increase gain.

\subsection*{PART D2: TCP Fast}

http://netlab.caltech.edu/publications/FAST-ToN-final-060209-2007.pdf

TCP Fast is a delay-based algorithm that periodically updates the window size
based on the average RTT [REF]. <INSERT UPDATE RULE>.

$$cwnd \leftarrow (1 - \gamma)* cwnd + \gamma \left(\frac{cwnd * {RT_{min}}}{RTT} + \alpha \right)$$

The RTT is estimated with the low pass filter function as priorly described, and
the RTTmin is a moving minimum. Additionally, our implementation does not limit
window growth to twice the window size to allow for more aggressive adaptation
to network changes. 

\subsection*{PART D3: TANGRA}

We further developed the idea of RTT-threshold based scheme by incorporating two
more ideas. We first make the assumption that the minimum propagation time does
not change over time. Given this assumption, our algorithm tracks the minimum
RTT and assumes this is the propagation time. Further, we assume that the delay
above the propagation time is proportional to the queueing delay. We write a
control-loop that adjusts the congestion window based on these two assumptions.

The second idea we had was to adjust the congestion window in a way that is
proportional to the difference between the current RTT estimate and a target
RTT. We observed that We define our target RTT to be * RTTmin, where α is an
adjustable parameter. Our best results are achieved with = 1.4.

\section*{Part E: Naming TANGRA}

Tangra is yet
ANother
Gruesome
Recursive
Acronym
\end{document}
